\documentclass[12pt,]{report}
\usepackage{lmodern}
\usepackage{amssymb,amsmath}
\usepackage{ifxetex,ifluatex}
\usepackage{fixltx2e} % provides \textsubscript
\ifnum 0\ifxetex 1\fi\ifluatex 1\fi=0 % if pdftex
  \usepackage[T1]{fontenc}
  \usepackage[utf8]{inputenc}
\else % if luatex or xelatex
  \ifxetex
    \usepackage{mathspec}
  \else
    \usepackage{fontspec}
  \fi
  \defaultfontfeatures{Ligatures=TeX,Scale=MatchLowercase}
\fi
% use upquote if available, for straight quotes in verbatim environments
\IfFileExists{upquote.sty}{\usepackage{upquote}}{}
% use microtype if available
\IfFileExists{microtype.sty}{%
\usepackage{microtype}
\UseMicrotypeSet[protrusion]{basicmath} % disable protrusion for tt fonts
}{}
\usepackage[margin=1in]{geometry}
\usepackage{hyperref}
\hypersetup{unicode=true,
            pdftitle={Open Science Toolbox},
            pdfauthor={Lutz Heil; lheil@students.uni-mainz.de; LMU Open Science Center},
            pdfborder={0 0 0},
            breaklinks=true}
\urlstyle{same}  % don't use monospace font for urls
\usepackage{longtable,booktabs}
\usepackage{graphicx,grffile}
\makeatletter
\def\maxwidth{\ifdim\Gin@nat@width>\linewidth\linewidth\else\Gin@nat@width\fi}
\def\maxheight{\ifdim\Gin@nat@height>\textheight\textheight\else\Gin@nat@height\fi}
\makeatother
% Scale images if necessary, so that they will not overflow the page
% margins by default, and it is still possible to overwrite the defaults
% using explicit options in \includegraphics[width, height, ...]{}
\setkeys{Gin}{width=\maxwidth,height=\maxheight,keepaspectratio}
\IfFileExists{parskip.sty}{%
\usepackage{parskip}
}{% else
\setlength{\parindent}{0pt}
\setlength{\parskip}{6pt plus 2pt minus 1pt}
}
\setlength{\emergencystretch}{3em}  % prevent overfull lines
\providecommand{\tightlist}{%
  \setlength{\itemsep}{0pt}\setlength{\parskip}{0pt}}
\setcounter{secnumdepth}{5}

%%% Use protect on footnotes to avoid problems with footnotes in titles
\let\rmarkdownfootnote\footnote%
\def\footnote{\protect\rmarkdownfootnote}

%%% Change title format to be more compact
\usepackage{titling}

% Create subtitle command for use in maketitle
\newcommand{\subtitle}[1]{
  \posttitle{
    \begin{center}\large#1\end{center}
    }
}

\setlength{\droptitle}{-2em}

  \title{Open Science Toolbox}
    \pretitle{\vspace{\droptitle}\centering\huge}
  \posttitle{\par}
  \subtitle{A Collection of Resources to Facilitate Open Science Practices}
  \author{Lutz Heil \\ \href{mailto:lheil@students.uni-mainz.de}{\nolinkurl{lheil@students.uni-mainz.de}} \\ LMU Open Science Center}
    \preauthor{\centering\large\emph}
  \postauthor{\par}
      \predate{\centering\large\emph}
  \postdate{\par}
    \date{April 14, 2018}

\usepackage{titlesec}
\titleformat{\chapter}
  {\normalfont\LARGE\bfseries}{\thechapter}{1em}{}
\titlespacing*{\chapter}{0pt}{3.5ex plus 1ex minus .2ex}{2.3ex plus .2ex}
\usepackage{booktabs}
\usepackage{amsthm}
\usepackage{xcolor}% http://ctan.org/pkg/xcolor
\usepackage{hyperref}% http://ctan.org/pkg/hyperref
\hypersetup{
  colorlinks=true,
  linkcolor=blue,
  urlcolor=blue
}
\usepackage{titling}
\pretitle{\begin{center}\huge \vskip 3em}
\posttitle{%
  \par\large \vskip 0.5em
  A Collection of Resources to Facilitate Open Science Practices
  \par\end{center}\vskip 1em}
\preauthor{%
  \begin{center}
  \large \lineskip 0.5em%
  \itshape
  \begin{tabular}[t]{c}
  }
\postauthor{\end{tabular}\par\end{center}}
\predate{\begin{center}\large}
\postdate{%
  \normalsize
  \par
  [last edit: \today ]
  \vfill
  \includegraphics[width=4cm,height=6cm]{images/lmu-osc_logo_small.jpg}
  \end{center}
}
\makeatletter
\def\thm@space@setup{%
  \thm@preskip=8pt plus 2pt minus 4pt
  \thm@postskip=\thm@preskip
}
\makeatother
\renewcommand\thesubsection{}
\renewcommand\thesubsubsection{}

\begin{document}
\maketitle

{
\setcounter{tocdepth}{1}
\tableofcontents
}
\chapter*{Preface}\label{preface}
\addcontentsline{toc}{chapter}{Preface}

The motivation to engage in a more transparent research and the actual
implementation of Open Science practices into the research workflow can
be two very distinct things. Work on the motivational side has been
successfully done in various contexts (see section
\protect\hyperlink{key_papers}{\emph{Key Papers}}). Here the focus
rather lays on closing the intention-action gap. Providing an overview
of handy Open Science tools and resources might facilitate the
implementation of these practices and thereby foster Open Science on a
practical level rather than on a theoretical.

So far, there is a vast body of helpful tools that can be used in order
to foster Open Science practices. Without doubt, all of these tools add
value to a more transparent research. But for reasons of clarity and to
save time which would be consumed by trawling through the web, this
toolbox aims at providing only a selection of links to these resources
and tools. Our goal is to give a short overview on possibilities of how
to enhance your Open Science practices without consuming too much of
your time.

\textbf{How can I use the toolbox?} If you're time is too valuable to
check out the toolbox entirely, a first start is to read the first
chapter \protect\hyperlink{getting_started}{\emph{Getting Started}}.
Though, the main body of the toolbox consists of the
\protect\hyperlink{res_researchers}{\emph{Resources for Researchers}}.
Here you'll find Open Science resources ordered by the stage of the
research cycle. If you are interested in teaching about Open Science or
if you search for workshop material to educate yourself, you might want
to jump to chapter 3: \protect\hyperlink{res_teaching}{\emph{Resources
for Teaching}}. Finally there is a small chapter called
\protect\hyperlink{community}{\emph{Community}} which can be helpful if
you seek information on how to get involved and where to get Open
Science news.

\textbf{Context.} This toolbox has been created during my work at the
\href{http://www.osc.uni-muenchen.de/index.html}{\textbf{Open Science
Center (OSC)}} of the Ludwig-Maximilians-Universität, Munich. This
document is supposed to serve as a merger of all resources that are
provided in the online toolbox (available on:
\url{http://www.osc.uni-muenchen.de/toolbox/index.html}). Special thanks
go to my supervisor
\href{http://www.osc.uni-muenchen.de/members/individual-members/schoenbrodt/index.html}{Felix
Schönbrodt} who came up with the idea and made important contributions.

\textbf{Contribute.} The toolbox raises no claim to completeness, it
should rather serve as selection of tools. If you think there is (1) an
important tool or article that should be included, (2) a broken link or
citation, or (3) any other issue or question, please feel free to
contact me and contribute to this toolbox:
\href{mailto:lheil@students.uni-mainz.de}{\nolinkurl{lheil@students.uni-mainz.de}}

\hypertarget{getting_started}{\chapter{Getting
Started}\label{getting_started}}

If you don't know where to start, here are 10 easy steps from
\href{https://osf.io/hjx5p/}{Felix Schönbrodt's Open Science workshop},
that you can take towards a more transparent and reproducible research
practice. Every step will add a little more scientific value to your
research.

\begin{enumerate}
\def\labelenumi{\arabic{enumi}.}
\tightlist
\item
  \textbf{Create an account on OSF} (\url{http://osf.io/})
\item
  \textbf{Upload the material for an existing study} (questionnaires,
  maybe reproducible analysis scripts) to an OSF project.
\item
  \textbf{Add an open license to all of your figures} (so that you can
  reuse them in later publications, blog posts, or presentations:
  „Figure available under a CC-BY4.0 license at osf.io/XXXX.``
\item
  For the next project: \textbf{Change the consent forms} in a way that
  \textbf{open data} would be possible for that project (see
  \url{https://osf.io/mgwk8/wiki/Consent\%20form\%20templates\%20for\%20open\%20data/}).
\item
  \textbf{Sign the PRO initiative} and expect openness (or a
  justification why not) if you review another paper
  (\url{https://opennessinitiative.org/})
\item
  For the next data analysis: Practice to create \textbf{scripts for
  reproducible data analysis} (e.g., SPSS syntax, R scripts). All
  analytic steps that lead from raw data to the final results should be
  reproducible.
\item
  \textbf{Let a master student preregister his/her thesis}. Can be
  either a „local preregisteration``, or a proper preregistration at OSF
  or at \url{https://aspredicted.org/}. See this workshop material for
  how to do a preregistration: \url{https://osf.io/yd487/}
  \url{https://osf.io/mx7yp/}
\item
  \textbf{Do you own first preregistration}; enter the Prereg challenge
  and get 1000\$: \url{https://cos.io/prereg/}
\item
  \textbf{Publish your first open data set}: Ensure anonymity, provide a
  codebook. See here for details:
  \url{http://econtent.hogrefe.com/doi/pdf/10.1026/0033-3042/a000341}
\item
  Team up with colleagues and \textbf{establish a local open science
  initiative} (enter your name and affiliation in
  \href{https://osf.io/tbkzh/wiki/Wer\%20macht\%20bei\%20meiner\%20Uni\%20mit\%3F/}{this
  list} and see other colleagues that want to engage in a local
  initiative)
\end{enumerate}

\begin{center}\rule{0.5\linewidth}{\linethickness}\end{center}

\begin{itemize}
\tightlist
\item
  \url{http://www.osc.uni-muenchen.de/news/os_flyer/index.html} This
  info flyer gives a great overview on easy steps to take toward
  transparent research.
\item
  \href{https://felixhenninger.gitbooks.io/open-science-knowledge-base/content/}{Felix
  Henninger's Open Science knowledge base} is a further helpful resource
  for quick answers to basic Open Science questions
\end{itemize}

\hypertarget{res_researchers}{\chapter{Resources for
Researchers}\label{res_researchers}}

In this chapter we provide a collection of resources and tools for
researchers to facilitate Open Science practices. The resources are
further ordered by stage of the research cycle
(\protect\hyperlink{plan}{\emph{Plan your Study}},
\protect\hyperlink{conduct}{\emph{Conduct your Study}},
\protect\hyperlink{analyse}{\emph{Analyse your Data}},
\protect\hyperlink{publish}{\emph{Publish your Data, Materials, and
Paper}}). Also, a good first source of answers to Open Science questions
is
\href{https://felixhenninger.gitbooks.io/open-science-knowledge-base/content/}{Felix
Henninger's Open Science FAQ}.

\hypertarget{plan}{\section{Plan your Study}\label{plan}}

Before diving into the data collection process, it might feel like Open
Science practices such as pre-registration are consuming a lot of
valuable time. But apart from the importance of these practices for
credibility and transparency of your research, these practices can
actually help organizing your project and saving time\ldots{} especially
when you don't need to search for resources.

\subsection{Evaluate Research}\label{evaluate-research}

This process should actually be part of the literature search before
even planing your study. A quick way to evaluate the research that you
found during your search is to use the following tools. They enable you
to find discussions, evaluations, peer reviews, and replication reports
on specific articles:

\begin{itemize}
\tightlist
\item
  \url{https://www.altmetric.com/}
\item
  \url{http://curatescience.org/}
\item
  \url{https://pubpeer.com/} (also available as browser add-on, and
  integrated in the Altmetric's score)
\end{itemize}

\subsection{Checklist for Research
Workflow}\label{checklist-for-research-workflow}

Throughout the research cycle you can use Brian A. Nosek's 8-step
\href{https://osf.io/mv8pj/wiki/home/}{Checklist for Research Workflow}
to control for an efficient open science research workflow.

\subsection{Consent Forms}\label{consent-forms}

When data will be shared publicly particiants need to be informed about
their anonymized data being shared. Here are some full german templates
of consent forms for \href{https://osf.io/3d5xb/}{dyadic experiments}
and \href{https://osf.io/kv37u/}{non-dyadic experiments} (collected by
the OSC). For english expressions that can be employed in consent forms,
see the \href{https://osf.io/g4jfv/wiki/home/}{COS Reproducible Research
and Statistics Training's collection}.

\subsection{Power Analysis}\label{power-analysis}

\begin{itemize}
\tightlist
\item
  \href{https://osf.io/adkj4/}{COS Reproducible Research and Statistics
  Training} offers a vast body of resources and material on Power
  Analysis, e.g.~an \href{https://osf.io/asf53/}{Introduction to Power
  Analysis} by Courtney Soderberg and R code examples for conducting
  power analyses
\item
  \href{https://osf.io/d76gc/}{Advanced Power Analysis Workshop} by
  Felix Schönbrodt
\end{itemize}

\subsection{Registered Reports (RR)}\label{registered-reports-rr}

The Center for Open Science (COS) gives detailed explanations and
resources on Registered Reports as well as a list of journals that use
Registered Reports as publishing format: \url{https://cos.io/rr/}. Links
to the repsective guideline are provided.

Examplary guidelines for authors and reviewers:
\url{https://osf.io/pukzy/}

\subsection{Pre-Registration}\label{pre-registration}

\begin{itemize}
\tightlist
\item
  A
  \href{http://blogs.worldbank.org/impactevaluations/a-pre-analysis-plan-checklist}{pre-analysis
  checklist} by David McKenzie
\item
  A
  \href{http://cega.berkeley.edu/assets/cega_events/92/Pre-Analysis_Plan_Template_Alejandro_Ganimian.pdf}{pre-analysis
  template} provided by Alejandro Ganimian
\item
  Pre-registration platform on the \textbf{Open Science Framework}:
  \url{https://osf.io/prereg/}
\item
  \href{http://help.osf.io/m/registrations/l/546603-enter-the-preregistration-challenge}{OSF
  Guide to pre-registration}
\end{itemize}

\textbf{\emph{Other Registries for Pre-Registration}}

\begin{itemize}
\tightlist
\item
  \url{https://aspredicted.org/}
\item
  \href{https://www.socialscienceregistry.org/}{AEA RCT Registry}: The
  American Economic Association's registry for randomized controlled
  trials in fields of economics an other social sciences
\item
  \href{http://ridie.3ieimpact.org/}{RIDIE Registry}: The Registry for
  International Development Impact Evaluations (open to experimental and
  observational studies assessing the impact of development programs
\item
  \href{http://egap.org/content/registration}{EGAP Registry}: Evidence
  in Governance and Politics is providing an unsupervised stopgap
  function to store designs until the creation of a general registry for
  social science research
\end{itemize}

\subsection{DataWiz}\label{datawiz}

The Leibniz Institute of Psychology Information (ZDIP) offers an
automated assistant tool for researchers to document and manage their
data from the beginning of the research cycle onwards. It includes
assistance in creating data management plans and sharing data
(internally as well as externally to the community). The tool, called
\href{https://datawiz.leibniz-psychology.org/DataWiz/}{\textbf{DataWiz}},
is available in english and german. It is further attached to the
\href{https://datawizkb.leibniz-psychology.org/}{DataWiz knowledge base}
which contains helpful information, guidelines, tools and resources for
different steps in the research cycle

\begin{center}\rule{0.5\linewidth}{\linethickness}\end{center}

\hypertarget{conduct}{\section{Conduct your Study}\label{conduct}}

During data collection, using open software, the Open Science Framework,
and other collaboration tools can not only enhance your workflow and
collaboration in the research team but also create a foundation for data
sharing and reproducibility. Here we selected a few resources for this
stage.

\subsection{Open Software}\label{open-software}

\begin{itemize}
\tightlist
\item
  \href{http://www.psychopy.org/}{PsychoPy}: Open-source software as a
  free tool to run neuroscience or psychology experiments written in
  Python. Represents an alternative to Presentation™ or Inquisit™.
\item
  \href{https://www.psytoolkit.org/}{PsyToolkit}: Open-source software
  for running online or lab-based experiments (psychology).
\item
  \href{http://osdoc.cogsci.nl/}{OpenSesame}: Open-source tool for
  building experiments for psychology, neuroscience, and experimental
  economics. It provides an easy graphical interface for beginners and
  Python scripting for advanced users. Open Sesame can be integrated in
  the Open Science Framework.
\item
  \href{https://labjs.felixhenninger.com/}{lab.js}: created by Felix
  Henninger. A new open tool with graphical interface for building
  browser-based experiments.
\end{itemize}

\subsection{Born Open Data}\label{born-open-data}

Data that is automatically uploaded to a repository (e.g.~GitHub)
including time stamps and automatically generated logs is called
\emph{born open}. Advantages can be full openness \& transparency, data
management solutions, and simplification of data sharing. For further
information see:

\begin{itemize}
\tightlist
\item
  Rouder, J. N. (2016). The what, why, and how of born-open data.
  \emph{Behavior research methods, 48}(3), 1062-1069.
  \href{https://link.springer.com/content/pdf/10.3758\%2Fs13428-015-0630-z.pdf}{https://link.springer.com/article/10.3758/s13428-015-0630-z}
\item
  Rouder, J., Haaf, J. M., \& Snyder, H. K. (2018, March 25). Minimizing
  Mistakes In Psychological Science.
  \url{http://doi.org/10.17605/OSF.IO/GXCY5}
\end{itemize}

\subsection{Open Science Framework
(OSF)}\label{open-science-framework-osf}

The \href{https://osf.io/}{OSF} serves as data repository and
collaboration tool. Researchers can structure their projects and
controll access to their stored data. A version control system and a
live-editing mode enhance workflow between collaborators. OSF also
implements an add-on function which enables the user to sync a project
with other tools and services such as Dataverse, Figshare, GitHub,
Zotero. \href{https://cos.io/our-products/osf/}{Here} is an overview on
OSF's feature set and its use in every step of the research cycle.

For a detailed tutorial on using the OSF to share research products,
see:

\begin{itemize}
\tightlist
\item
  Soderberg, C. K. (2018). Using OSF to Share Data: A Step-by-Step
  Guide. Advances in Methods and Practices in Psychological Science.
  Advance online publication. DOI:
  \url{https://doi.org/10.1177/2515245918757689}
\end{itemize}

\subsection{Other Collaboration Tools}\label{other-collaboration-tools}

\begin{itemize}
\tightlist
\item
  \href{https://figshare.com/}{Figshare}
\item
  \href{https://datadryad.org/}{Dryad}
\item
  \href{https://zenodo.org/}{Zenodo}
\item
  \href{https://www.github.com}{GitHub}
\item
  \href{https://www.overleaf.com/}{Overleaf}
\item
  \href{https://paperhive.org/}{PaperHive}
\end{itemize}

\subsection{Data Anonymization}\label{data-anonymization}

\begin{itemize}
\tightlist
\item
  \href{https://www.openaire.eu/}{The OpenAIRE2020 project} developed a
  web-based anonymization application that helps ensuring participants'
  privacy rights. A respective online tool is in beta version at this
  point: \url{https://amnesia.openaire.eu/index.html}
\item
  Alternative: \href{http://arx.deidentifier.org/}{ARX Data
  Anonymization Tool}
\item
  See Ruben Arslan's workshop on
  \href{https://osf.io/9j27d/}{Maintaining privacy with open data}
\end{itemize}

\begin{center}\rule{0.5\linewidth}{\linethickness}\end{center}

\hypertarget{analyse}{\section{Analyse your Data}\label{analyse}}

For your research to be transparent and reproducible, a vital part is to
provide data cleaning instructions and analysis code (ideally by using
nonproprietary software). See the
\href{https://opennessinitiative.org/making-your-analyses-public/}{PRO
Initiative's basic guidelines for making your analyses
publi}\href{https://opennessinitiative.org/making-your-analyses-public/}{c}
for a few information. This collection of resources aims at providing
some helpful links to facilitate and improving your analysis sharing
practices.

\subsection{R}\label{r}

Analysing your data with free software such as
\href{https://www.r-project.org/}{R} enhances reproducibility without
the limitations of proprietary software. A common way to use R is by
writing analysis code and graphics code in the
\href{https://www.rstudio.com/}{RStudio} interface. It further
implements \href{https://rmarkdown.rstudio.com/}{R Markdown}, with can
be used to create documents, reports, and presentations that are fully
reproducible.

Here are some helpful links for researchers who want to learn R:

\begin{itemize}
\tightlist
\item
  RStudio links to different online guides to learning R:
  \url{https://www.rstudio.com/online-learning/}
\item
  \href{http://swirlstats.com/}{Swirl} is an interactive learning tool
  which can be directly embedded in RStudio.
\item
  RStudio cheat sheets:
  \url{https://www.rstudio.com/resources/cheatsheets/}
\item
  Grolemund \& Wickham's \href{http://r4ds.had.co.nz/}{\emph{R for Data
  Science}}
\item
  Hadley Wickham's
  \href{http://vita.had.co.nz/papers/tidy-data.pdf}{\emph{Tidy Data}}
\end{itemize}

\hypertarget{repro_research}{\subsection{Reproducible
Research}\label{repro_research}}

The \emph{PRO Initiative} gives a few basic guidelines for authors on
how to facilitate reproducibility when sharing your analyses:
\url{https://opennessinitiative.org/making-your-analyses-public/}

More specific collections of useful advice on this topic can be found in
the following sources:

\subsubsection{Code}\label{code}

\begin{itemize}
\tightlist
\item
  \href{http://home.bi.no/charlotte.ostergaard/students/CodeAndData.pdf}{Code
  and Data for the Social Sciences: A Practitioner's Guide} by Matthew
  Gentzkow \& Jesse M. Shapiro
\item
  \href{http://swcarpentry.github.io/r-novice-gapminder/}{R for
  Reproducible Scientific Analysis}: A course by Thomas Wright and
  Naupaka Zimmerman
\item
  \href{https://gupsych.github.io/research_cycle/}{The Reseach Cycle}
  course on principles on reproducible research and practical training
  in statistical programming with R, taught by Dale Barr and Lisa
  DeBruine
\item
  If you want to make your code citable, see this GitHub Guide:
  \url{https://guides.github.com/activities/citable-code/}
\end{itemize}

\subsubsection{Metadata}\label{metadata}

\begin{itemize}
\tightlist
\item
  \href{https://www.projecttier.org/tier-protocol/dress-protocol/}{The
  DRESS Protocall} by Project TIER describes what the final
  documentation of your study should consist of (for empirical social
  sciences).
\item
  \href{https://rubenarslan.github.io/codebook/}{Codebook Cookbook} by
  Ruben Arslan: R-package (and online-tool) to create a codebook for
  your dataset.
\item
  Creating a codebook within SPSS:
  \url{https://libguides.library.kent.edu/SPSS/Codebooks}
\end{itemize}

\subsubsection{\texorpdfstring{``Works on my Machine''
Error}{Works on my Machine Error}}\label{works-on-my-machine-error}

In his paper (available
\href{https://doi.org/10.1017/S1049096516000196}{here}), Nicholas Eubank
makes the case for increasing reproducibility by testing files on a
different computer. Testing or even sharing code via cloud-based
platforms prevents deficits in reproducibility that occur when code runs
on the researcher's local platform but not on others'. Avoid the so
called \emph{WOMME} by using tools like:

\begin{itemize}
\tightlist
\item
  RStudio Cloud (\url{https://rstudio.cloud/})
\item
  Code Ocean (\url{https://codeocean.com/})
\end{itemize}

\subsection{Workflow Documentation}\label{workflow-documentation}

\begin{itemize}
\tightlist
\item
  Rouder, Haaf, and Snyder (2018) wrote a helpful tutorial on how to
  organize a lab in the face of Open Science practices:
  \url{https://psyarxiv.com/gxcy5}
\item
  \href{https://rmarkdown.rstudio.com/}{R Markdown}: Create fully
  reproducible documents that combine code execution and documentation.
  A big advantage is the variety of output format: documents (e.g.~Word,
  PDF, interactive R notebook, HTML), presentation slides, shiny apps,
  websites and more. Multiple languages including R, python, and SQL can
  be used.
\item
  \href{https://github.com}{GitHub} serves as data repository and active
  research workflow tool. Tracking of contributions of others enables
  version control on your files. This tool is especially useful for a
  research team that collaborates on developing code.
\item
  As one of multiple features, \href{https://osf.io}{Open Science
  Framework}'s version control system and its live-editing mode
  facilitate collaboration within the research team.
\end{itemize}

\subsection{p-Hacking}\label{p-hacking}

\begin{itemize}
\tightlist
\item
  \emph{\href{https://osf.io/u4jgz/}{p-Hack like a pro}} by Felix
  Schönbrodt
\item
  \href{https://osf.io/9fx5q/}{\emph{Do's and Don'ts of Data Analysis}}
  by Felix Schönbrodt
\item
  An overview on \textbf{questionable research practices} (QRP) by
  Ulrich Schimmack can be found
  \href{https://replicationindex.wordpress.com/2015/01/24/questionable-research-practices-definition-detect-and-recommendations-for-better-practices/}{here}.
\item
  Check out: \url{http://shinyapps.org/apps/p-hacker/} for an
  interactive tool on
  p-hacking\href{http://shinyapps.org/apps/p-hacker/}{}
\end{itemize}

\subsection{Jupyter}\label{jupyter}

\href{https://jupyter.org/}{The Jupyter Notebook} is an open source
web-application for interactive computing. Virtual notebooks support
over 40 programming languages, can be shared, collaboratively edited and
can return interactive output. Uses vary from data cleaning,
transformation and visualization to machine learning, statistical
modeling and more.

\href{https://www.datacamp.com/community/tutorials/tutorial-jupyter-notebook}{A
tutorial on the Jupyter Notebook} by Karlijn Willems is a first
place-to-go for trying out the Jupyter Notebook. Also, help can be
seeked at the Jupyter community (e.g.~on
\href{https://github.com/jupyter/help}{GitHub} or
\href{https://stackoverflow.com/questions/tagged/jupyter}{StackOverflow})

\begin{center}\rule{0.5\linewidth}{\linethickness}\end{center}

\hypertarget{publish}{\section{Publish your Data, Material, and
Paper}\label{publish}}

In the publication stage of the research cycle, things can get a little
tricky when you want to embrace Open Science values. The following links
will help you find Open Access journals, check your publisher's
policies, and guide you through licencing issues. Also, resources on
sharing of data and materials are given.

\subsection{FAIR Principles}\label{fair-principles}

Before Sharing your data, your material, and your paper, we recommend to
check out the
\href{https://www.force11.org/group/fairgroup/fairprinciples}{FAIR Data
Principles}, that aim to make your data:

\begin{itemize}
\tightlist
\item
  Findable
\item
  Accessible
\item
  Interoperable
\item
  Re-usable
\end{itemize}

Also see our section
\protect\hyperlink{repro_research}{\emph{Reproducible Research}} for
further resurces on the topic of re-usabilty principles

\subsection{Open Data Repositories}\label{open-data-repositories}

\href{https://www.re3data.org/}{Global Registry of Research Data
Repositories} covering data repositories from different academic
disciplines. You can directly search for repositories or browse the
registry by several filters.

Examples of common general-purpose repositories:

\begin{itemize}
\tightlist
\item
  \href{https://osf.io/}{Open Science Framework}
\item
  \href{https://datadryad.org/}{Dryad Digital Repository}
\item
  \href{https://figshare.com/}{Figshare}
\item
  \href{https://dataverse.harvard.edu/}{Harvard Dataverse Network}
\item
  \href{https://zenodo.org/}{Zenodo}
\end{itemize}

A list of general-purpose and domain-specific data repositories is also
provided by Masuzzo P, Martens L. (2017) Do you speak open science?
Resources and tips to learn the language. \emph{PeerJ Preprints}
5:e2689v1 \url{https://doi.org/10.7287/peerj.preprints.2689v1}

\subsection{How To Find Open Access
Journals}\label{how-to-find-open-access-journals}

\begin{itemize}
\tightlist
\item
  The \href{https://doaj.org/}{Directory of Open Access Journals} serves
  as directory for researchers who are looking for high-quality,
  peer-reviewed Open Access journals and respective articles. You will
  also find useful information on publication charges and licencing
  policies as well as links to editorial information.
\item
  Use the \href{http://cofactorscience.com/journal-selector}{CoFactor
  Journal Seleceting Tool} to filter for journals fitting your search
  criteria (e.g.~Open Access).
\item
  The Eigenfactor Project provides a list of no-fee Open Access journals
  for all fields:
  \url{http://www.eigenfactor.org/openaccess/fullfree.php}
\end{itemize}

\subsection{Publisher's Policies?}\label{publishers-policies}

\href{http://www.sherpa.ac.uk/romeo/index.php}{Sherpa/RoMEO} is a
database containing publishers' policies on self-archiving of journal
articles. Since policies vary between publishers this tool will shed
some light if you are unsure about publisher's policies.

\subsection{Share Preprints}\label{share-preprints}

Although depending on publisher's policies, it is often possible to
publish preprints of your paper. A common preprint repository is
\href{https://ArXiv.org}{Cornell University's ArXiv} which serves in
many research fields.

For specific domains, here are some examples of preprint service (mostly
powered by the \href{https://osf.io/preprints/}{Open Science Framework
Preprints}):

\begin{itemize}
\tightlist
\item
  Psychological Sciences:
  \href{https://osf.io/preprints/psyarxiv/}{PsyArXiv}
\item
  Social Sciences: \href{https://osf.io/preprints/socarxiv}{SocARXIV}
\item
  Engineering: \href{https://osf.io/preprints/engrxiv}{engrXiV}
\item
  Agriculture and Allied Sciences:
  \href{https://osf.io/preprints/agrixiv}{AgriXiv}
\end{itemize}

\subsection{Licencing}\label{licencing}

The \emph{Peer Reviewers' Openness Initiative (PRO)} gives a very
helpful overview over licensing issues that are prevalent when
publishing your paper, data, and materials:
\url{https://opennessinitiative.org/licensing-issues/}:

\begin{quote}
``When thinking about what one can, must, or should do with Open
Material and Open Data, one has to differentiate on the one hand
legally-enforceable rules (which are handled with legal licenses) and,
on the other hand, rules and standards of the scientific community. If a
piece of work is in the public domain (e.g., a CC0 license) there is no
legal requirement to give attribution, but as a scientist one still has
the ethical obligation to give a proper citation. Laws, ethics, and
professional courtesy are all ways that the community can protect those
that open their data.''
\end{quote}

Also, the \href{http://opendefinition.org/licenses/}{OpenDefinition}
project lists licenses for content and data.

\hypertarget{res_teaching}{\chapter{Resources for
Teaching}\label{res_teaching}}

If you are interested in teaching Open Science practices, see this
selection of curriculum material, complete examples of online courses,
and further tools to implement in an Open Science workshop.

\subsection{Open Science Workshop Material of the LMU Open Science
Center}\label{open-science-workshop-material-of-the-lmu-open-science-center}

\href{https://osf.io/zjrhu/}{}

Angelika Stefan, Felix Schönbrodt, and Lena Schiestel created materials
for workshops on Open Science related topics:

\begin{itemize}
\tightlist
\item
  Open Science Introduction
\item
  Preregistration
\item
  Power Analysis
\item
  Open Data / Open Materials / Privacy
\item
  Open Access
\end{itemize}

You can use or remix the material under a CC-BY licence for your
teaching purposes.

\subsection{Open and Reproducible Research Trainer
Space}\label{open-and-reproducible-research-trainer-space}

The COS provides an \href{https://osf.io/qsb2c/}{Open and Reproducible
Research Trainer Space} which contains a vast body of material for Open
Science workshops including sections on workshop planning (pre workshop
checklists, flyers, email tempates, etc.) and curriculum (slides, full
manual).

\subsection{Open Science Training
Handbook}\label{open-science-training-handbook}

Theoretical Open Science basics, organizational aspects, examples,
practical guidance, and a further resources for each topic are provided
by the
\href{https://open-science-training-handbook.gitbooks.io/book/content/}{Open
Science Training Handbook}. This is a great source for Open Science
Trainer (and those who are planing on becoming one).

\subsection{Examples of
Workshops/MOOCs}\label{examples-of-workshopsmoocs}

\begin{itemize}
\tightlist
\item
  \textbf{\emph{Making an Impact with Open Science by Dr.~Michiel de
  Jong}}
\end{itemize}

This course focuses on the benefits researchers can reap from following
Open Science principles. After giving theoretical introductions to basic
concepts and advantages of Open Science, the course also aims at giving
practical training concerning data sharing, open access publishing and
the use of social media. It is organized in four modules containing both
video lectures and readings. Study load is around 10 hrs.

\url{https://ocw.tudelft.nl/courses/making-impact-open-science/}

\begin{itemize}
\tightlist
\item
  \textbf{\emph{Open Science MOOC}}
\end{itemize}

\begin{quote}
``This MOOC is designed to help equip students and researchers with the
skills they need to excel in a modern research environment. It brings
together the efforts and resources of hundreds of researchers and
practitioners who have all dedicated their time and experience to create
a community platform to help propel research forward. The content of
this MOOC is distilled into 10 core modules. Each module will comprise a
complete range of resources including videos, research articles, dummy
datasets and code, as well as `homework' tasks to complete as
individuals. Because you don't learn how to do Open Science by reading;
you learn by doing it.''
\end{quote}

\url{https://opensciencemooc.eu/}

This course also offers a wide module-related
\href{https://opensciencemooc.eu/open-science-resources/}{collection of
resources}.

\begin{itemize}
\tightlist
\item
  \textbf{\emph{Transparent and Open Social Science Research}}
\end{itemize}

5 week (à 4 hours/week) full course by Berkeley Initiative for
Transparency in the Social Sciences (BITSS) on the FutureLearn platform.

\url{https://www.futurelearn.com/courses/open-social-science-research}

\begin{itemize}
\tightlist
\item
  \textbf{\emph{Teaching Reproducible Data Analysis in R}}
\end{itemize}

In March 2017, the Institute of Neuroscience and Psychology and the
School of Psychology at the University of Glasgow hosted a one-day
workshop on teaching reproducible data analysis using R. The workshop
curriculum and slides are openly available. Also, a further resource
page links to additional material.

\url{https://gupsych.github.io/trdair_workshop/index.html}

\begin{itemize}
\tightlist
\item
  \textbf{\emph{BITSS Research Transparency and Reproducibility Training
  (RT2)}}
\end{itemize}

All public resources of this workshop holding place in Amsterdam in
April 2018 are provided. It aims at giving an overview of tools and best
practices in transparent and reproducible social sciences.

\url{https://osf.io/fw28g/}

\subsection{Interactive Tools}\label{interactive-tools}

\href{https://gupsych.github.io/trdair_workshop/index.html}{http://shinyapps.org/apps/p-hacker/}

\hypertarget{key_papers}{\chapter{Key Papers}\label{key_papers}}

This chapter should serve as a recommondation list of Open Science
articles, guidelines, and other toolboxes offering helpful resources.

\textbf{Transparency Guide}

Klein, O., Hardwicke, T. E., Aust, F., Breuer, J., Danielsson,
H.,\ldots{} Frank, M. C. (2018, March 25). A practical guide for
transparency in psychological science. Retrieved from
\href{http://psyarxiv.com/rtygm}{psyarxiv.com/rtygm}

\textbf{Open Science: What, Why, and How}

Spellman, B., Gilbert, E. A., \& Corker, K. S. (2017, September 20).
Open Science: What, Why, and How. Retrieved from
\href{http://psyarxiv.com/ak6jr}{psyarxiv.com/ak6jr}

\textbf{PRO Initiatives Guidelines}

``These pages provide guidance in implementing open practices and links
to resources on the web that help authors contribute to the scientific
literature in a more open way.''
\url{https://opennessinitiative.org/guidelines-for-authors/}

\textbf{Open Science FAQ}

\href{https://felixhenninger.gitbooks.io/open-science-knowledge-base/content/}{Felix
Henninger's FAQ to Open Science} topics offers a great resource for
quick answers concerning Open Science topcis

\textbf{Managing and Sharing Data: Best Practice for Researchers}

\href{http://www.data-archive.ac.uk/media/2894/managingsharing.pdf}{UK
Data Archive's guide} for managing data for sharing and for ethics,
consent, and licensing issues.

\textbf{DataWiz Knowledge Base}

Best practice guidelines, tools and resources for researchers before,
during and after data collection:
\url{https://datawizkb.leibniz-psychology.org/}

\textbf{Resource Library by BITSS}

Berkeley Initiative for Transparency in the Social Sciences:
\url{https://www.bitss.org/resource-tag/education/}

\textbf{Open Science MOOC Resource Library}

Great collection of resources as part of an Open Science MOOC
\url{https://opensciencemooc.eu/open-science-resources/}

\textbf{Further Recommended Readings}

\begin{itemize}
\tightlist
\item
  Reproducibility Project: Psychology: \url{https://osf.io/ezcuj/}
\item
  The Replication Network: \url{https://replicationnetwork.com}
\item
  Ioannidis, J. P. (2005). Why most published research findings are
  false. \emph{PLoS medicine, 2}(8), e124.
  \url{https://doi.org/10.1371/journal.pmed.0020124}
\item
  Simmons, J. P., Nelson, L. D., \& Simonsohn, U. (2011). False-positive
  psychology: Undisclosed flexibility in data collection and analysis
  allows presenting anything as significant. \emph{Psychological
  science, 22}(11)\emph{,} 1359-1366. Available at:
  \url{http://journals.sagepub.com/doi/pdf/10.1177/0956797611417632}
\item
  Simmons, Joseph P. and Nelson, Leif D. and Simonsohn, Uri, A 21 Word
  Solution (October 14, 2012). Available at SSRN:
  \url{https://ssrn.com/abstract=2160588} or
  \url{http://dx.doi.org/10.2139/ssrn.2160588}
\item
  Masuzzo P, Martens L. (2017) Do you speak open science? Resources and
  tips to learn the language. \emph{PeerJ Preprints} 5:e2689v1
  \url{https://doi.org/10.7287/peerj.preprints.2689v1}
\item
  Garret Christensen's
  \href{https://github.com/garretchristensen/BestPracticesManual/blob/master/Manual.pdf}{Manual
  of Best Practices in Transparent Social Science}
\end{itemize}

\hypertarget{community}{\chapter{Community}\label{community}}

\subsection{Other Open Science
Initiatives}\label{other-open-science-initiatives}

Information on other Open Science initiatives, resources of the Open
Science community and how to stay tuned on Open Science news.

\begin{itemize}
\tightlist
\item
  \href{https://cos.io/}{Center for Open Science}
\item
  \href{https://osf.io/tbkzh/}{Netzwerk der Open-Science-Initiativen
  (NOSI)}
\item
  \href{https://osf.io/tbkzh/wiki/home/}{List of Open Science
  Initiatives}
\item
  \href{https://www.bitss.org/}{Berkeley Initiative for Transparency in
  the Social Sciences}
\item
  \href{http://www.researchtransparency.org/}{Commitment to Research
  Transparency and Open Science}
\item
  \href{https://www.bihealth.org/de/quest-center/}{QUEST Center for
  Transforming Biomedical Research}
\item
  \href{https://www.mcgill.ca/neuro/open-science-0}{Open Science
  Initiative in NeuroScience at McGill University, Canada}
\end{itemize}

French national action plan for buiding an Open Science ecosystem:
\url{https://www.etalab.gouv.fr/wp-content/uploads/2018/04/PlanOGP-FR-2018-2020-VF-FR.pdf}
(pp.57 / Engagement 18)

\subsection{Social Media}\label{social-media}

Twitter:

\begin{itemize}
\tightlist
\item
  \href{https://twitter.com/openscience}{@OpenScience}
\end{itemize}

Facebook:

\begin{itemize}
\tightlist
\item
  \href{https://www.facebook.com/groups/psychmap/about/}{PsychMAP}
\item
  \href{https://www.facebook.com/groups/853552931365745/?ref=group_header}{PsychMAD}
\end{itemize}

Blogs:

\begin{itemize}
\tightlist
\item
  \url{http://www.the100.ci/}
\item
  \url{http://sometimesimwrong.typepad.com/}
\item
  \href{https://cos.io/blog/}{Center for Open Science Blog}
\end{itemize}

Other formats:

\begin{itemize}
\tightlist
\item
  \url{http://www.openscienceradio.org/}
\item
  \url{https://ask-open-science.org/}
\end{itemize}


\end{document}
